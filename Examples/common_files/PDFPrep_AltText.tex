Alternative text, or `Alt text', is a textual description of an equation, link or figure that can be used to replace the visual information in that element. This is often seen as a text `pop-up' in PDF readers. For example, passing the pointer over the following equation should reveal a pop-up:

\begin{equation}
\pdftooltip{a^2+b^2=c^2}{An equation}
\end{equation}

Alt text can be added after the PDF is compiled using a PDF editor such as Adobe's Acrobat Pro. Alternatively -- and probably best for ensuring that the final document is what the author intended -- it can be generated from within the source document using the \texttt{pdftooltip} environment from the \texttt{pdfcomment} package. The previous equation was generated using \verb?\pdftooltip{a^2+b^2=c^2}{An equation}?.

The same approach can be used to create alt text for images. For example, Figure \ref{fig:NRELimagesWithAltText} has been labeled with a tool tip. 

\begin{figure*}
	\centering
        \begin{subfigure}[t]{.45\linewidth}
		\centering
		{\pdftooltip{\includegraphics[height=2in]{../common_files/21206.jpg}}{Wind turbines at the Forward Wind Energy Center in Fond du Lac and Dodge Counties, Wisconsin. (Photo by Ruth Baranowski / NREL)}}
		\caption{Wind turbines at the Forward Wind Energy Center in Fond du Lac and Dodge Counties, Wisconsin. (Photo by Ruth Baranowski / NREL)}\label{fig:21206WithAltText}
	\end{subfigure}%
        \hfill
        \begin{subfigure}[t]{.45\linewidth}
		\centering
		{\pdftooltip{\includegraphics[height=2in]{../common_files/20018.jpg}}{Aerial view of the National Wind Technology Center. (Photo by Dennis Schroeder / NREL)}}
		\caption{Aerial view of the National Wind Technology Center. (Photo by Dennis Schroeder / NREL)}\label{fig:20018WithAltText}
	\end{subfigure}
	\caption{Images with Alt Text}\label{fig:NRELimagesWithAltText}
\end{figure*}